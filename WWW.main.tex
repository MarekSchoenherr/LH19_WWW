%= local definitions of macros ============================
\newcommand{\Herwig}{H\protect\scalebox{0.8}{ERWIG}\xspace}
\newcommand{\Pythia}{P\protect\scalebox{0.8}{YTHIA}\xspace}
\newcommand{\Sherpa}{S\protect\scalebox{0.8}{HERPA}\xspace}
\newcommand{\Rivet}{R\protect\scalebox{0.8}{IVET}\xspace}
\newcommand{\Recola}{R\protect\scalebox{0.8}{ECOLA}\xspace}
\newcommand{\Amegic}{A\protect\scalebox{0.8}{MEGIC}\xspace}
\newcommand{\Professor}{P\protect\scalebox{0.8}{ROFESSOR}\xspace}
\newcommand{\eps}{\varepsilon}
\newcommand{\mc}[1]{\mathcal{#1}}
\newcommand{\mr}[1]{\mathrm{#1}}
\newcommand{\mb}[1]{\mathbb{#1}}
\newcommand{\tm}[1]{\scalebox{0.95}{$#1$}}
\newcommand{\vp}{\ensuremath{\vphantom{\int_a^b}}}
\newcommand{\vP}{\ensuremath{\vphantom{\int\limits_a^b}}}

%= title + authors =====================================
\section{NLO corrections to off-shell \texorpdfstring{$WWW$}{WWW} production\protect\footnote{
  S.~Dittmaier,
  G.~Knippen,
  M.~Sch{\"o}nherr,
  C.~Schwan}{}}

%= MANDATORY label ======================================
\label{sec:WWW}

%= (optional) preamble ================================== 

%= intro ===== ==========================================
\subsection{Introduction}
\label{sec:WWW:intro}

ATLAS 8\,TeV analysis \cite{Aaboud:2016ftt}.


%= methods ===== ========================================
\subsection{Methods}
\label{sec:WWW:methods}

In this contribution, we are comparing the numerical results 
obtained by two different and independent calculations for 
off-shell $WWW$ production.
On the one hand side, we use a combination of \Sherpa 
\cite{Bothmann:2019yzt,Gleisberg:2008ta,Bothmann:2016nao,Hoeche:2014rya} 
and \Recola \cite{Actis:2012qn,Actis:2016mpe}, as presented 
in \cite{Schonherr:2018jva}. 
Therein, \Sherpa provides the tree-level matrix elements, 
infrared substraction, process management and phase-space 
integration through its matrix element generator \Amegic 
\cite{Krauss:2001iv,Gleisberg:2007md,Schonherr:2017qcj}. 
\Recola is interfaced \cite{Biedermann:2017yoi} to provide 
all renormalised virtual corrections.

On the other hand ...

All processes are calculated including all triple, double, 
single and non-resonant topologies, as well as all interferences 
between channels with different vector and Higgs boson intermediate
states.

%= comparison ===== =====================================
\subsection{Comparison and results}
\label{sec:WWW:comparison}

\begin{table}[t!]
  \centering
  \begin{tabular}{c|c}
    Cut & Value \\\hline
    $p_\mathrm{T}(\ell)$ & $[20,\infty]\,\text{GeV}$ \\
    $\eta(\ell)$ & $[-2.5,2.5]$ \\
    $p_\mathrm{T}(\ell_1)$ & $[27,\infty]\,\text{GeV}$ \\
    $\Delta R(\ell_i,\ell_j)$ & $[0.1,\infty]$
  \end{tabular}
  \caption{
    Definition of the fiducial region. Lepton requirements relate 
    to dressed leptons using a Cone algorithm with $\Delta R_\text{dress}=0.1$.
    \label{tab:WWW:cuts}
  }
\end{table}


\begin{table}[t!]
  \centering
  (i) $\mathrm{pp}\to e^-\mu^+\tau^+\bar{\nu}_e\nu_\mu\nu_\tau+X$\\
  \begin{tabular}{l|c|c|c|c|c}
    \hline
    13\,TeV \vP
    & LO [fb] & NLO [fb] 
    & $\Delta_{q\bar{q}}^\text{EW}$ [\%]
    & $\Delta_{q\gamma}^\text{EW}$ [\%]
    & $\Delta^\text{QCD} [\%]$\\\hline
    \hfill DKS \vp
    & 0.194990(19) & 0.2626(10) & -7.70(40) & 7.220(5) & 38.02(04) \\
    \hfill\Sherpa{}+\Recola \vp
    & 0.195118(83) & 0.2649(21) & -7.38(57) & 7.217(3) & 38.11(10) \\\hline
  \end{tabular}\\[2mm]
  (ii) $\mathrm{pp}\to e^+\mu^-\tau^-\nu_e\bar{\nu}_\mu\bar{\nu}_\tau+X$\\
  \begin{tabular}{l|c|c|c|c|c}
    \hline
    13\,TeV \vP
    & LO [fb] & NLO [fb] 
    & $\Delta_{q\bar{q}}^\text{EW}$ [\%]
    & $\Delta_{q\gamma}^\text{EW}$ [\%]
    & $\Delta^\text{QCD} [\%]$\\\hline
    \hfill DKS \vp
    & 0.118411(12) & 0.1597(06) & -7.00(30) & 7.260(5) & 37.17(4) \\
    \hfill\Sherpa{}+\Recola \vp
    & 0.118420(73) & 0.1584(14) & -6.73(51) & 7.267(3) & 37.07(9) \\\hline
  \end{tabular}
  \caption{
    Comparison of cross sections at 13\,TeV.
    \label{tab:WWW:xsecs13}
  }
\end{table}

\begin{table}[t!]
  \centering
  (i) $\mathrm{pp}\to e^-\mu^+\tau^+\bar{\nu}_e\nu_\mu\nu_\tau+X$\\
  \begin{tabular}{l|c|c|c|c|c}
    \hline
    14\,TeV \vP
    & LO [fb] & NLO [fb] 
    & $\Delta_{q\bar{q}}^\text{EW}$ [\%]
    & $\Delta_{q\gamma}^\text{EW}$ [\%]
    & $\Delta^\text{QCD} [\%]$\\\hline
    \hfill DKS \vp
    & 0.209820(20) & 0.2872(12) & -7.80(40) & 7.780(5) & 40.04(04) \\
    \hfill\Sherpa{}+\Recola \vp
    & 0.209962(85) & 0.2898(23) & -7.47(59) & 7.793(4) & 40.10(11) \\\hline
  \end{tabular}\\[2mm]
  (ii) $\mathrm{pp}\to e^+\mu^-\tau^-\nu_e\bar{\nu}_\mu\bar{\nu}_\tau+X$\\
  \begin{tabular}{l|c|c|c|c|c}
    \hline
    14\,TeV \vP
    & LO [fb] & NLO [fb] 
    & $\Delta_{q\bar{q}}^\text{EW}$ [\%]
    & $\Delta_{q\gamma}^\text{EW}$ [\%]
    & $\Delta^\text{QCD} [\%]$\\\hline
    \hfill DKS \vp
    & 0.129986(13) & 0.1779(07) & -7.20(40) & 7.730(5) & 39.15(04) \\
    \hfill\Sherpa{}+\Recola \vp
    & 0.130016(76) & 0.1766(15) & -6.81(55) & 7.738(4) & 39.18(10) \\\hline
  \end{tabular}
  \caption{
    Comparison of cross sections at 14\,TeV.
    \label{tab:WWW:xsecs13}
  }
\end{table}

In order to compare the calculations of the independent 
calculations laid out in the previous section we 
choose the following setup.
We compute the production cross sections for the production of 
final states with (at least) one electron, one muon and one 
$\tau$-lepton with the corresponding three neutrinos at the 
LHC 13 and 14\,TeV at leading and next-to-leading order 
in both the QCD and electroweak coupling of the Standard Model
in the phase space detailed in 
Tab.\ \ref{tab:WWW:cuts}. 
The gauge boson masses and widths are defined by their on-shell 
value provided by the Particle Data Group \cite{}, 
\begin{center}
  \begin{tabular}{ll}
    $M_W^\text{OS}=80.379\,\text{GeV}$\qquad & $\Gamma_W^\text{OS}=2.085\,\text{GeV}$ \\
    $M_Z^\text{OS}=91.1876\,\text{GeV}$\qquad & $\Gamma_Z^\text{OS}=2.4952\,\text{GeV}$.
  \end{tabular}
\end{center}
They are then converted to pole masses using 
\begin{equation}
  M_V=\frac{M_V^\text{OS}}{\sqrt{1+\left(\frac{\Gamma_V^\text{OS}}{M_V^\text{OS}}\right)^2}}\;,
  \qquad
  \Gamma_V=\frac{\Gamma_V^\text{OS}}{\sqrt{1+\left(\frac{\Gamma_V^\text{OS}}{M_V^\text{OS}}\right)^2}}
\end{equation}
In addtion, we set the Higgs boson mass and top-quark masses 
and widths to 
\begin{center}
  \begin{tabular}{ll}
    $M_h=125\,\text{GeV}$ & $\Gamma_h=0.004088\,\text{GeV}$ \\
    $M_t=173\,\text{GeV}$ & $\Gamma_t=0$\;. \\
  \end{tabular}
\end{center}
The bottom and all other quarks and leptons (including the $\tau$) 
are considered massless.
We perform our calculation in the complex mass scheme 
\cite{Denner:2005fg,Denner:2014zga},
defining
\begin{equation}
  \mu_V^2=M_V^2-iM_V\Gamma_V\;.
\end{equation}
The CKM matrix is parametrised using the Cabibbo angle 
\begin{equation}
  \theta_\text{C}=0.22731\;,\nonumber
\end{equation}
neglecting mixing with the third generation. 
All parameters of the electroweak part of the Standard Model 
are fixed using the $G_\mu$-scheme \cite{} with
\begin{equation}
  G_\mu=1.1663787\cdot 10^{-5}\,\text{GeV}^{-2}\;.\nonumber
\end{equation}
We then define the electromagnetic coupling through the 
real parts of the complex masses, i.e.\
\begin{equation}
  \alpha=\frac{\sqrt{2}}{\pi}\,G_\mu\,M_W\left(1-\frac{M_W^2}{M_Z^2}\right);.
\end{equation}
The electroweak parameters are renormalised in the same scheme.





%= methods ===== ========================================
\subsection{Conclusions}
\label{sec:WWW:conclusions}



%= undefine macros (MANDATORY) ====================
\let\Herwig\undefined
\let\Pythia\undefined
\let\Sherpa\undefined
\let\Rivet\undefined
\let\Recola\undefined
\let\Professor\undefined
\let\Amegic\undefined
\let\eps\undefined
\let\mc\undefined
\let\mr\undefined
\let\mb\undefined
\let\tm\undefined
\let\vp\undefined
\let\vP\undefined




