%= local definitions of macros ============================
\newcommand{\Herwig}{H\protect\scalebox{0.8}{ERWIG}\xspace}
\newcommand{\Pythia}{P\protect\scalebox{0.8}{YTHIA}\xspace}
\newcommand{\Sherpa}{S\protect\scalebox{0.8}{HERPA}\xspace}
\newcommand{\Rivet}{R\protect\scalebox{0.8}{IVET}\xspace}
\newcommand{\Recola}{R\protect\scalebox{0.8}{ECOLA}\xspace}
\newcommand{\Amegic}{A\protect\scalebox{0.8}{MEGIC}\xspace}
\newcommand{\Professor}{P\protect\scalebox{0.8}{ROFESSOR}\xspace}
\newcommand{\OpenLoops}{O\protect\scalebox{0.8}{PENLOOPS 2}\xspace}
\newcommand{\Collier}{C\protect\scalebox{0.8}{OLLIER}\xspace}
\newcommand{\Madgraph}{M\protect\scalebox{0.8}{G5\_aMC@NLO}\xspace}
\newcommand{\eps}{\varepsilon}
\newcommand{\mc}[1]{\mathcal{#1}}
\newcommand{\mr}[1]{\mathrm{#1}}
\newcommand{\mb}[1]{\mathbb{#1}}
\newcommand{\tm}[1]{\scalebox{0.95}{$#1$}}
\newcommand{\vp}{\ensuremath{\vphantom{\int_a^b}}}
\newcommand{\vP}{\ensuremath{\vphantom{\int\limits_a^b}}}

%= title + authors =====================================
\section{NLO corrections to off-shell $WWW$ production\protect\footnote{
  S.~Dittmaier,
  G.~Knippen,
  M.~Sch{\"o}nherr,
  C.~Schwan}{}}

%= MANDATORY label ======================================
\label{sec:WWW}

%= (optional) preamble ================================== 

%= intro ===== ==========================================
\subsection{Introduction}
\label{sec:WWW:intro}

To cite a paper, use bibtex:

Examples: Les Houches 2015 report~\cite{Badger:2016bpw},
paper~\cite{Aaboud:2016ffv}, tHV~\cite{'tHooft:1972fi}, and again~\cite{'tHooft:1972fi}


%= methods ===== ========================================
\subsection{Methods}
\label{sec:WWW:methods}

In this contribution, we are comparing the numerical results 
obtained by two different and independent calculations for 
off-shell $WWW$ production.
On the one hand side, we use a combination of \Sherpa \cite{} 
and \Recola \cite{Actis:2012qn,Actis:2016mpe}, as presented in \cite{}.
Therein, \Sherpa provides the tree-level matrix elements, 
infrared substraction, process management and phase-space 
integration through its matrix element generator \Amegic 
\cite{}. 
\Recola is interfaced \cite{} to provide all renormalised 
virtual corrections.

On the other hand, the results denoted as DKS are the ones presented in Ref.~\cite{Dittmaier:2019twg}.
These numbers were calculated and double-checked using two private codes; the first one was developed specifically for this process, and the second one is a generic code that was already used to calculate other EW processes~\cite{Ballestrero:2018anz,Denner:2019tmn,Denner:2019zfp}.
The first code uses dipole subtraction as presented in Ref.~\cite{Catani:1996vz} for QCD corrections and Refs.~\cite{Dittmaier:1999mb,Dittmaier:2008md} for EW corrections.
Matrix elements were generated using \Madgraph \cite{Alwall:2014hca} and \Recola.
The second code is able to use either dipole subtraction as presented in Ref.~\cite{Catani:1996vz} for both QCD and EW corrections, the latter using the trivial substitutions for the Casimir operator presented e.g.\ in section 3.2 of Ref.~\cite{Kallweit:2014xda}, or, alternatively, the dipole subtraction methods used in the first code.
The matrix elements are provided by \OpenLoops \cite{Cascioli:2011va,Kallweit:2014xda,Buccioni:2019sur}, which uses \Collier \cite{Denner:2016kdg,Denner:2002ii,Denner:2005nn,Denner:2010tr} for the evaluation of rank 0 and 1 tensor one-loop integrals.
Both codes use multi-channel Monte Carlo techniques \cite{Hilgart:1992xu,Kleiss:1994qy} for the phase-space integration with phase-space maps similar to the ones presented in Ref.~\cite{Dittmaier:2002ap}.

%= comparison ===== =====================================
\subsection{Comparison and results}
\label{sec:WWW:comparison}

\begin{table}[t!]
  \centering
  (i) $\mathrm{pp}\to e^-\mu^+\tau^+\bar{\nu}_e\nu_\mu\nu_\tau+X$\\
  \begin{tabular}{l|c|c|c|c|c}
    \hline
    13\,TeV \vP
    & LO [fb] & NLO [fb] 
    & $\Delta_{q\bar{q}}^\text{EW}$ [\%]
    & $\Delta_{q\gamma}^\text{EW}$ [\%]
    & $\Delta^\text{QCD} [\%]$\\\hline
    \hfill DKS \vp
    & 0.194990(19) & 0.2626(10) & -7.70(40) & 7.220(5) & 38.02(04) \\
    \hfill\Sherpa{}+\Recola \vp
    & 0.195118(83) & 0.2649(21) & -7.38(57) & 7.217(3) & 38.11(10) \\\hline
  \end{tabular}\\[2mm]
  (ii) $\mathrm{pp}\to e^+\mu^-\tau^-\nu_e\bar{\nu}_\mu\bar{\nu}_\tau+X$\\
  \begin{tabular}{l|c|c|c|c|c}
    \hline
    13\,TeV \vP
    & LO [fb] & NLO [fb] 
    & $\Delta_{q\bar{q}}^\text{EW}$ [\%]
    & $\Delta_{q\gamma}^\text{EW}$ [\%]
    & $\Delta^\text{QCD} [\%]$\\\hline
    \hfill DKS \vp
    & 0.118411(12) & 0.1597(06) & -7.00(30) & 7.260(5) & 37.17(4) \\
    \hfill\Sherpa{}+\Recola \vp
    & 0.118420(73) & 0.1584(14) & -6.73(51) & 7.267(3) & 37.07(9) \\\hline
  \end{tabular}
  \caption{
    Comparison of cross sections at 13\,TeV.
    \label{tab:WWW:xsecs13}
  }
\end{table}

\begin{table}[t!]
  \centering
  (i) $\mathrm{pp}\to e^-\mu^+\tau^+\bar{\nu}_e\nu_\mu\nu_\tau+X$\\
  \begin{tabular}{l|c|c|c|c|c}
    \hline
    14\,TeV \vP
    & LO [fb] & NLO [fb] 
    & $\Delta_{q\bar{q}}^\text{EW}$ [\%]
    & $\Delta_{q\gamma}^\text{EW}$ [\%]
    & $\Delta^\text{QCD} [\%]$\\\hline
    \hfill DKS \vp
    & 0.209820(20) & 0.2872(12) & -7.80(40) & 7.780(5) & 40.04(04) \\
    \hfill\Sherpa{}+\Recola \vp
    & 0.209962(85) & 0.2898(23) & -7.47(59) & 7.793(4) & 40.10(11) \\\hline
  \end{tabular}\\[2mm]
  (ii) $\mathrm{pp}\to e^+\mu^-\tau^-\nu_e\bar{\nu}_\mu\bar{\nu}_\tau+X$\\
  \begin{tabular}{l|c|c|c|c|c}
    \hline
    14\,TeV \vP
    & LO [fb] & NLO [fb] 
    & $\Delta_{q\bar{q}}^\text{EW}$ [\%]
    & $\Delta_{q\gamma}^\text{EW}$ [\%]
    & $\Delta^\text{QCD} [\%]$\\\hline
    \hfill DKS \vp
    & 0.129986(13) & 0.1779(07) & -7.20(40) & 7.730(5) & 39.15(04) \\
    \hfill\Sherpa{}+\Recola \vp
    & 0.130016(76) & 0.1766(15) & -6.81(55) & 7.738(4) & 39.18(10) \\\hline
  \end{tabular}
  \caption{
    Comparison of cross sections at 14\,TeV.
    \label{tab:WWW:xsecs13}
  }
\end{table}



%= methods ===== ========================================
\subsection{Conclusions}
\label{sec:WWW:conclusions}



%= undefine macros (MANDATORY) ====================
\let\Herwig\undefined
\let\Pythia\undefined
\let\Sherpa\undefined
\let\Rivet\undefined
\let\Recola\undefined
\let\Professor\undefined
\let\Amegic\undefined
\let\OpenLoops\undefined
\let\Collier\undefined
\let\eps\undefined
\let\mc\undefined
\let\mr\undefined
\let\mb\undefined
\let\tm\undefined
\let\vp\undefined
\let\vP\undefined




